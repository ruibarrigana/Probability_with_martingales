

\documentclass[11pt]{article} % use larger type; default would be 10pt
\usepackage[utf8]{inputenc} % set input encoding (not needed with XeLaTeX)

\usepackage{geometry} % to change the page dimensions
\geometry{a4paper} % or letterpaper (US) or a5paper or....


%%% PACKAGES
\usepackage{graphicx}
\usepackage{amsmath}
\usepackage[english]{babel}
\usepackage{natbib} 
\usepackage{amsfonts}
\usepackage{hyphenat}
\usepackage{booktabs} % for much better looking tables
\usepackage{enumerate}
\usepackage{kbordermatrix} % for matrices with indexes and brackets instead of parentheses;
\usepackage{array} % for better arrays (eg matrices) in maths
\usepackage{paralist} % very flexible & customisable lists (eg. enumerate/itemize, etc.)
\usepackage[group-separator={,}]{siunitx}
\usepackage{verbatim} % adds environment for commenting out blocks of text & for better verbatim
\usepackage{subfig} % make it possible to include more than one captioned figure/table in a single float
% These packages are all incorporated in the memoir class to one degree or another...
\usepackage{caption}
\usepackage{sidecap}
\usepackage{setspace}
\sidecaptionvpos{figure}{c}

%%% HEADERS & FOOTERS
\usepackage{fancyhdr} % This should be set AFTER setting up the page geometry
\pagestyle{fancy} % options: empty , plain , fancy
\renewcommand{\headrulewidth}{0pt} % customise the layout...
\lhead{}\chead{}\rhead{}
\lfoot{}\cfoot{\thepage}\rfoot{}

%%% SECTION TITLE APPEARANCE
\usepackage{sectsty}
%\allsectionsfont{\sffamily\mdseries\upshape} % (See the fntguide.pdf for font help)
% (This matches ConTeXt defaults)

%%% ToC (table of contents) APPEARANCE
\usepackage[nottoc,notlof,notlot]{tocbibind} % Put the bibliography in the ToC
\usepackage[titles,subfigure]{tocloft} % Alter the style of the Table of Contents
\renewcommand{\cftsecfont}{\rmfamily\mdseries\upshape}
\renewcommand{\cftsecpagefont}{\rmfamily\mdseries\upshape} % No bold!
\captionsetup[figure]{labelfont={bf,it},textfont={it}}
\setlength{\parindent}{0cm}
%%% END Article customizations

%%% The "real" document content comes below...

\title{Probability with martingales - exercises}
\author{}
\date{} % Activate to display a given date or no date (if empty),
         % otherwise the current date is printed 

\begin{document}
{\bf \maketitle \par}


\section*{E10.4}

\vspace{\baselineskip}
First part (predictability).

I assume $S$ and $T$ are again stopping times with respect to $\left(\Omega, \mathcal{F}, \left\lbrace \mathcal{F}_{n}\right\rbrace \right)$ (as in the previous exercise). The function $\mathrm{1}_{\left(S,T\right]}\left(n,\omega\right)$ is $\mathcal{F}_{n-1}$-measurable, since: \\

$\left\lbrace \omega : \mathrm{1}_{\left(S,T\right]}\left(n,\omega\right)=0\right\rbrace=\left\lbrace \omega : S(\omega) > n-1 \right\rbrace \cup \left\lbrace \omega : T(\omega) \leq n-1\right\rbrace \in \mathcal{F}_{n-1} \quad ;$\\

$\left\lbrace \omega : \mathrm{1}_{\left(S,T\right]}\left(n,\omega\right)=1\right\rbrace=\left\lbrace \omega : \mathrm{1}_{\left(S,T\right]}\left(n,\omega\right)=0\right\rbrace^{c} \in \mathcal{F}_{n-1} \quad .$\\
\vspace{\baselineskip}
 
Second part.\\ 
 
If $X$ is a supermartingale and $\mathrm{1}_{\left(S,T\right]}$ is previsible, $\left( \mathrm{1}_{\left(S,T\right]}\bullet X \right)$ is also a supermartingale and is null at zero (see page 97). From section 10.3 (in particular the bit which uses the tower property of conditional expectations),\\

$\mathrm{E}\left[\left(\mathrm{1}_{\left(S,T\right]}\bullet X \right)_{n} \right]=\mathrm{E}\left[\left(\mathrm{1}_{\left(S,T\right]}\bullet X \right)_{n} |\mathcal{F}_{0}\right]\leq \left(\mathrm{1}_{\left(S,T\right]}\bullet X \right)_{0}=0$. \\

Since

$X_{T\wedge n}=X_{S \wedge n}+\left(\mathrm{1}_{\left(S,T\right]}\bullet X \right)_{n}$\\

and\\

$\mathrm{E}\left[X_{T\wedge n}\right]=\mathrm{E}\left[X_{S \wedge n}\right]+\mathrm{E}\left[\left(\mathrm{1}_{\left(S,T\right]}\bullet X \right)_{n}\right]$,\\

the result is proven.


\clearpage
\end{document}
