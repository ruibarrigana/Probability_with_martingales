% !TEX TS-program = pdflatex
% !TEX encoding = UTF-8 Unicode

% This is a simple template for a LaTeX document using the "article" class.
% See "book", "report", "letter" for other types of document.

\documentclass[11pt]{article} % use larger type; default would be 10pt
\usepackage[utf8]{inputenc} % set input encoding (not needed with XeLaTeX)

%%% Examples of Article customizations
% These packages are optional, depending whether you want the features they provide.
% See the LaTeX Companion or other references for full information.

%%% PAGE DIMENSIONS
\usepackage{geometry} % to change the page dimensions
\geometry{a4paper} % or letterpaper (US) or a5paper or....
% \geometry{margin=2in} % for example, change the margins to 2 inches all round
% \geometry{landscape} % set up the page for landscape
%   read geometry.pdf for detailed page layout information

 % support the \includegraphics command and options

% \usepackage[parfill]{parskip} % Activate to begin paragraphs with an empty line rather than an indent

%%% PACKAGES
\usepackage{graphicx}
\usepackage{amsmath}
\usepackage[english]{babel}
\usepackage{natbib} 
\usepackage{amsfonts}
\usepackage{hyphenat}
\usepackage{booktabs} % for much better looking tables
\usepackage{enumerate}
\usepackage{kbordermatrix} % for matrices with indexes and brackets instead of parentheses;
\usepackage{array} % for better arrays (eg matrices) in maths
\usepackage{paralist} % very flexible & customisable lists (eg. enumerate/itemize, etc.)
\usepackage[group-separator={,}]{siunitx}
\usepackage{verbatim} % adds environment for commenting out blocks of text & for better verbatim
\usepackage{subfig} % make it possible to include more than one captioned figure/table in a single float
% These packages are all incorporated in the memoir class to one degree or another...
\usepackage{caption}
\usepackage{sidecap}
\usepackage{setspace}
\sidecaptionvpos{figure}{c}

%%% HEADERS & FOOTERS
\usepackage{fancyhdr} % This should be set AFTER setting up the page geometry
\pagestyle{fancy} % options: empty , plain , fancy
\renewcommand{\headrulewidth}{0pt} % customise the layout...
\lhead{}\chead{}\rhead{}
\lfoot{}\cfoot{\thepage}\rfoot{}

%%% SECTION TITLE APPEARANCE
\usepackage{sectsty}
%\allsectionsfont{\sffamily\mdseries\upshape} % (See the fntguide.pdf for font help)
% (This matches ConTeXt defaults)

%%% ToC (table of contents) APPEARANCE
\usepackage[nottoc,notlof,notlot]{tocbibind} % Put the bibliography in the ToC
\usepackage[titles,subfigure]{tocloft} % Alter the style of the Table of Contents
\renewcommand{\cftsecfont}{\rmfamily\mdseries\upshape}
\renewcommand{\cftsecpagefont}{\rmfamily\mdseries\upshape} % No bold!
\captionsetup[figure]{labelfont={bf,it},textfont={it}}
\setlength{\parindent}{0cm}
%%% END Article customizations

%%% The "real" document content comes below...

\title{Probability with martingales - exercises}
\author{}
\date{} % Activate to display a given date or no date (if empty),
         % otherwise the current date is printed 

\begin{document}
{\bf \maketitle \par}


\section*{E10.3}

\vspace{\baselineskip}

If $S$ and $T$ are stopping times with respect to the filtration $\left\lbrace \mathcal{F}_{n} \right\rbrace$, then $\left\lbrace \omega : S\left(\omega\right) \leq n\right\rbrace \in \mathcal{F}_{n}$ and $\left\lbrace \omega : T\left(\omega\right) \leq n\right\rbrace \in \mathcal{F}_{n}$.

Due to the properties of $\sigma$-fields, 

$\left\lbrace \omega : \mathrm{min}\left[S\left(\omega\right),T\left(\omega\right)\right] \leq n\right\rbrace= \left\lbrace \omega : S\left(\omega\right) \leq n \right\rbrace \cup \left\lbrace \omega : T\left(\omega\right) \leq n \right\rbrace \in \left\lbrace \mathcal{F}_{n} \right\rbrace$.

\vspace{\baselineskip}

For the same reason,

$\left\lbrace \omega : \mathrm{max}\left[S\left(\omega\right),T\left(\omega\right)\right] \leq n\right\rbrace= \left\lbrace \omega : S\left(\omega\right) \leq n \right\rbrace \cap \left\lbrace \omega : T\left(\omega\right) \leq n \right\rbrace \in \left\lbrace \mathcal{F}_{n} \right\rbrace$.


\vspace{\baselineskip}
Also,
\begin{equation*}
\begin{array}{lcl}
\left\lbrace \omega : S(\omega)+T(\omega)=n\right\rbrace&=&\displaystyle \bigcup_{i=1}^{n-1}\left\lbrace \omega : S(\omega)=i, T(\omega)=n-i\right\rbrace\\
&=&\displaystyle \bigcup_{i=1}^{n-1}\left[\left\lbrace\ \omega : S(\omega)=i\right\rbrace \cap \left\lbrace T(\omega)=n-i\right\rbrace \right] \quad.
\end{array}
\end{equation*}

\vspace{\baselineskip}
Again, since $S$ is a stopping time with respect to the filtration $\left\lbrace \mathcal{F}_{n} \right\rbrace$, $\left\lbrace \omega : S(\omega)=i \right\rbrace \in \mathcal{F}_{i}$, for $i=1,2,... n-1$. But $\mathcal{F}_{i} \subseteq \mathcal{F}_{n} $, so $\left\lbrace \omega : S(\omega)=i \right\rbrace \in \mathcal{F}_{n}$. 

By the same reasoning, $\left\lbrace \omega : T(\omega)=n-i \right\rbrace \in \mathcal{F}_{n}$, for $i=1,2,... n-1$. \\

Hence $\left\lbrace \omega : S(\omega)=i \right\rbrace \cap \left\lbrace \omega : T(\omega)=n-i\right\rbrace \in \mathcal{F}_{n}$;\\ and $\displaystyle \bigcup_{i=1}^{n-1}\left[ \left\lbrace \omega : S(\omega)=i\right\rbrace \cap \left\lbrace \omega : T(\omega)=n-i\right\rbrace \right] \in \mathcal{F}_{n}$.

\clearpage
\bibliographystyle{Chicago}
\bibliography{Researchplan}
\end{document}
