% !TEX TS-program = pdflatex
% !TEX encoding = UTF-8 Unicode

% This is a simple template for a LaTeX document using the "article" class.
% See "book", "report", "letter" for other types of document.

\documentclass[11pt]{article} % use larger type; default would be 10pt
\usepackage[utf8]{inputenc} % set input encoding (not needed with XeLaTeX)

%%% Examples of Article customizations
% These packages are optional, depending whether you want the features they provide.
% See the LaTeX Companion or other references for full information.

%%% PAGE DIMENSIONS
\usepackage{geometry} % to change the page dimensions
\geometry{a4paper} % or letterpaper (US) or a5paper or....
% \geometry{margin=2in} % for example, change the margins to 2 inches all round
% \geometry{landscape} % set up the page for landscape
%   read geometry.pdf for detailed page layout information

 % support the \includegraphics command and options

% \usepackage[parfill]{parskip} % Activate to begin paragraphs with an empty line rather than an indent

%%% PACKAGES
\usepackage{graphicx}
\usepackage{amsmath}
\usepackage[english]{babel}
\usepackage{natbib} 
\usepackage{amsfonts}
\usepackage{hyphenat}
\usepackage{booktabs} % for much better looking tables
\usepackage{enumerate}
\usepackage{kbordermatrix} % for matrices with indexes and brackets instead of parentheses;
\usepackage{array} % for better arrays (eg matrices) in maths
\usepackage{paralist} % very flexible & customisable lists (eg. enumerate/itemize, etc.)
\usepackage[group-separator={,}]{siunitx}
\usepackage{verbatim} % adds environment for commenting out blocks of text & for better verbatim
\usepackage{subfig} % make it possible to include more than one captioned figure/table in a single float
% These packages are all incorporated in the memoir class to one degree or another...
\usepackage{caption}
\usepackage{sidecap}
\usepackage{setspace}
\sidecaptionvpos{figure}{c}

%%% HEADERS & FOOTERS
\usepackage{fancyhdr} % This should be set AFTER setting up the page geometry
\pagestyle{fancy} % options: empty , plain , fancy
\renewcommand{\headrulewidth}{0pt} % customise the layout...
\lhead{}\chead{}\rhead{}
\lfoot{}\cfoot{\thepage}\rfoot{}

%%% SECTION TITLE APPEARANCE
\usepackage{sectsty}
%\allsectionsfont{\sffamily\mdseries\upshape} % (See the fntguide.pdf for font help)
% (This matches ConTeXt defaults)

%%% ToC (table of contents) APPEARANCE
\usepackage[nottoc,notlof,notlot]{tocbibind} % Put the bibliography in the ToC
\usepackage[titles,subfigure]{tocloft} % Alter the style of the Table of Contents
\renewcommand{\cftsecfont}{\rmfamily\mdseries\upshape}
\renewcommand{\cftsecpagefont}{\rmfamily\mdseries\upshape} % No bold!
\captionsetup[figure]{labelfont={bf,it},textfont={it}}
\setlength{\parindent}{0cm}
%%% END Article customizations

%%% The "real" document content comes below...

\title{Probability with martingales - exercises}
\author{}
\date{} % Activate to display a given date or no date (if empty),
         % otherwise the current date is printed 

\begin{document}
{\bf \maketitle \par}


\section*{E10.2}


\vspace{\baselineskip}

$C_{n}:=cZ_{n-1}$, with $0<c<1$.

\vspace{\baselineskip}
\begin{equation*}
\begin{array}{lcl}
\mathrm{E}\left(\log Z_{n}|Z_{n-1}\right)&=&p \log\left[\left(1+c\right)Z_{n-1}\right]+q \log \left[\left(1-c\right)Z_{n-1}\right]\\
&=&\log Z_{n-1} + p \log \left(1+c\right)+q \log \left(1-c \right) \quad.
\end{array}
\end{equation*}

\begin{equation*}
\begin{array}{lll}
\frac{\mathrm{d}}{\mathrm{d}c}\mathrm{E}\left(\log Z_{n}|Z_{n-1}\right)&=&\frac{p}{1+c}-\frac{q}{1-c} 
\end{array}
\end{equation*}
and

\begin{equation*}
\begin{array}{lll}
\frac{p}{1+c}-\frac{q}{1-c} &=&0
\end{array}
\end{equation*}
at $c=2p-1$.

Also,
\begin{equation*}
\begin{array}{lll}
\frac{\mathrm{d}^{2}}{\mathrm{d}c^{2}}\mathrm{E}\left(\log Z_{n}|Z_{n-1}\right)&=&\frac{-p}{\left(1+c\right)^{2}}-\frac{q}{\left(1-c\right)^{2}} < 0 \quad.
\end{array}
\end{equation*}

So 
\begin{equation*}
\begin{array}{lcl}
\mathrm{E}\left(\log Z_{n}|Z_{n-1}\right)&\leq& \log \left(Z_{n-1} \right)+p \log \left(2p \right) + q \log \left[2\left(1-p\right)\right] \quad.
\end{array}
\end{equation*}

Now,
\begin{equation*}
\begin{array}{lcl}
\mathrm{E}\left[ \log Z_{n}-n\alpha|Z_{n-1}\right]&=&\mathrm{E}\left[\log Z_{n}|Z_{n-1}\right]-n\alpha\\
&\leq & \log \left(Z_{n-1}\right)+\log\left(2\right)+p\log \left(p\right) + q \log\left(q\right)-n\left[\log\left(2\right)+p\log \left(p\right) + q \log\left(q\right)\right] \\
&=&\log\left(Z_{n-1}\right)-\left(n-1\right)\alpha \quad.
\end{array}
\end{equation*}

Hence $\log\left(Z_{n} \right)-n\alpha$ is a supermartingale, and a martingale if $C_{n}= \left(2p-1\right)Z_{n-1}$ -- the best possible strategy. 

Being a supermartingale, it must be true, for all $n \geq 1$, that

\begin{equation*}
\begin{array}{lcl}
\mathrm{E}\left[\log \left(Z_{n}\right)-n\alpha \right] &\leq& \mathrm{E}\left[\log \left(Z_{0}\right)-0\alpha \right] \\
&=& \log \left(Z_{0}\right) 
\end{array} \quad.
\end{equation*}

But
\begin{equation*}
\begin{array}{rll}
\mathrm{E}\left[\log \left(Z_{n}\right)-n\alpha \right] &\leq& \log \left(Z_{0}\right)   \\
\Rightarrow \mathrm{E}\left[\log \left(Z_{n}\right)-\log \left( Z_{0}\right)\right]&\leq&n\alpha \\
\Rightarrow \mathrm{E}\left[\log \left(\frac{Z_{n}}{Z_{0}}\right)\right]&\leq&n\alpha\quad.
\end{array}
\end{equation*}




\clearpage
\bibliographystyle{Chicago}
\bibliography{Researchplan}
\end{document}
