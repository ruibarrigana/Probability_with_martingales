

\documentclass[11pt]{article} % use larger type; default would be 10pt
\usepackage[utf8]{inputenc} % set input encoding (not needed with XeLaTeX)

\usepackage{geometry} % to change the page dimensions
\geometry{a4paper} % or letterpaper (US) or a5paper or....


%%% PACKAGES
\usepackage{graphicx}
\usepackage{amsmath}
\usepackage[english]{babel}
\usepackage{natbib} 
\usepackage{amsfonts}
\usepackage{hyphenat}
\usepackage{booktabs} % for much better looking tables
\usepackage{enumerate}
%\usepackage{kbordermatrix} % for matrices with indexes and brackets instead of parentheses;
\usepackage{array} % for better arrays (eg matrices) in maths
\usepackage{paralist} % very flexible & customisable lists (eg. enumerate/itemize, etc.)
\usepackage[group-separator={,}]{siunitx}
\usepackage{verbatim} % adds environment for commenting out blocks of text & for better verbatim
\usepackage{subfig} % make it possible to include more than one captioned figure/table in a single float
% These packages are all incorporated in the memoir class to one degree or another...
\usepackage{caption}
\usepackage{sidecap}
\usepackage{setspace}
\sidecaptionvpos{figure}{c}

%%% HEADERS & FOOTERS
\usepackage{fancyhdr} % This should be set AFTER setting up the page geometry
\pagestyle{fancy} % options: empty , plain , fancy
\renewcommand{\headrulewidth}{0pt} % customise the layout...
\lhead{}\chead{}\rhead{}
\lfoot{}\cfoot{\thepage}\rfoot{}

%%% SECTION TITLE APPEARANCE
\usepackage{sectsty}
%\allsectionsfont{\sffamily\mdseries\upshape} % (See the fntguide.pdf for font help)
% (This matches ConTeXt defaults)

%%% ToC (table of contents) APPEARANCE
\usepackage[nottoc,notlof,notlot]{tocbibind} % Put the bibliography in the ToC
\usepackage[titles,subfigure]{tocloft} % Alter the style of the Table of Contents
\renewcommand{\cftsecfont}{\rmfamily\mdseries\upshape}
\renewcommand{\cftsecpagefont}{\rmfamily\mdseries\upshape} % No bold!
\captionsetup[figure]{labelfont={bf,it},textfont={it}}
\setlength{\parindent}{0cm}
%%% END Article customizations

%%% The "real" document content comes below...

\title{Probability with martingales - exercises}
\author{}
\date{} % Activate to display a given date or no date (if empty),
         % otherwise the current date is printed 

\begin{document}
{\bf \maketitle \par}


\section*{E10.6}

\vspace{\baselineskip}

Let $X_{n}$ be the total money at stake before the $\left(n+1\right)^{\mathrm{th}}$ chosen letter. \\
\begin{equation}
\begin{array}{lcl}
X_{n}&=&\displaystyle\sum_{a_{i} \in A} 26^{a_{i}}+\displaystyle\sum_{b_{i} \in B} 26^{b_{i}}+... \quad,
\end{array}
\end{equation} 
where each summation gives the total amount of money bet on letter A, B, etc. (the sets $A$, $B$,... will either be empty or contain -- non strictly --  positive integers).  


We have that\\
\begin{equation}
\begin{array}{lcl}
\mathrm{\textbf{E}}\left(X_{n+1}|X_{n}\right)&=&\displaystyle\frac{1}{26}\sum_{a_{i} \in A} 26^{a_{i}+1}+\frac{1}{26}\displaystyle\sum_{b_{i} \in B} 26^{b_{i}+1}+...+1 \\
&=&\displaystyle\sum_{a_{i} \in A} 26^{a_{i}}+\displaystyle\sum_{b_{i} \in B} 26^{b_{i}}+...+1\\
&=&X_{n}+1
\end{array}
\end{equation}
Using the tower property of conditional expectation repeatedly, \\
$\mathrm{\textbf{E}}\left(X_{n}\right)=n+1$, since $X_{0}\equiv 1 \quad.$

Also define, for all $n$:\\
$M_{n}:=X_{n}-(n+1)$, so that $\mathrm{\textbf{E}}\left(M_{n}\right)=0$, and\\
$C_{n}:=1\quad.$\\
\\
If the increments $M_{n}-M_{n-1}$ are bounded and $\mathrm{\textbf{E}}\left(T\right)< \infty$, then $\mathrm{\textbf{E}}\left(C \bullet M\right)_{T}=0$, by result 10.10.c. But $\left(C \bullet M\right)=M_{T}$, since $M_{0}=0$, and $M_{T}=X_{T}-\left(T+1\right)$. At the same time, it must be true that $X_{T}=26^{11}+26^{4}+26+1$ (which is the fortune of the 3 players in the game just after ABRACADABRA has been typed, plus the 1 pound that a new player bets just before time $T+1$). So $0=\mathrm{\textbf{E}}\left(C \bullet M\right)=\mathrm{\textbf{E}}\left[X_{T}-\left(T+1\right] \right)=\mathrm{\textbf{E}}\left[26^{11}+26^{4}+26+1-\left(T+1\right)\right]=26^{11}+26^{4}+26-\mathrm{\textbf{E}}\left(T\right)\quad.$\\
\\
It remains to prove that the increments $M_{n}-M_{n-1}=X_{n}-X_{n-1}-1$ are bounded and $\mathrm{\textbf{E}}\left(T\right)< \infty$.






\clearpage
\end{document}
