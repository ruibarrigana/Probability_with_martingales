% !TEX TS-program = pdflatex
% !TEX encoding = UTF-8 Unicode

% This is a simple template for a LaTeX document using the "article" class.
% See "book", "report", "letter" for other types of document.

\documentclass[11pt]{article} % use larger type; default would be 10pt
\usepackage[utf8]{inputenc} % set input encoding (not needed with XeLaTeX)

%%% Examples of Article customizations
% These packages are optional, depending whether you want the features they provide.
% See the LaTeX Companion or other references for full information.

%%% PAGE DIMENSIONS
\usepackage{geometry} % to change the page dimensions
\geometry{a4paper} % or letterpaper (US) or a5paper or....
% \geometry{margin=2in} % for example, change the margins to 2 inches all round
% \geometry{landscape} % set up the page for landscape
%   read geometry.pdf for detailed page layout information

 % support the \includegraphics command and options

% \usepackage[parfill]{parskip} % Activate to begin paragraphs with an empty line rather than an indent

%%% PACKAGES
\usepackage{graphicx}
\usepackage{amsmath}
\usepackage[english]{babel}
\usepackage{natbib} 
\usepackage{amsfonts}
\usepackage{hyphenat}
\usepackage{booktabs} % for much better looking tables
\usepackage{enumerate}
\usepackage{kbordermatrix} % for matrices with indexes and brackets instead of parentheses;
\usepackage{array} % for better arrays (eg matrices) in maths
\usepackage{paralist} % very flexible & customisable lists (eg. enumerate/itemize, etc.)
\usepackage[group-separator={,}]{siunitx}
\usepackage{verbatim} % adds environment for commenting out blocks of text & for better verbatim
\usepackage{subfig} % make it possible to include more than one captioned figure/table in a single float
% These packages are all incorporated in the memoir class to one degree or another...
\usepackage{caption}
\usepackage{sidecap}
\usepackage{setspace}
\sidecaptionvpos{figure}{c}

%%% HEADERS & FOOTERS
\usepackage{fancyhdr} % This should be set AFTER setting up the page geometry
\pagestyle{fancy} % options: empty , plain , fancy
\renewcommand{\headrulewidth}{0pt} % customise the layout...
\lhead{}\chead{}\rhead{}
\lfoot{}\cfoot{\thepage}\rfoot{}

%%% SECTION TITLE APPEARANCE
\usepackage{sectsty}
%\allsectionsfont{\sffamily\mdseries\upshape} % (See the fntguide.pdf for font help)
% (This matches ConTeXt defaults)

%%% ToC (table of contents) APPEARANCE
\usepackage[nottoc,notlof,notlot]{tocbibind} % Put the bibliography in the ToC
\usepackage[titles,subfigure]{tocloft} % Alter the style of the Table of Contents
\renewcommand{\cftsecfont}{\rmfamily\mdseries\upshape}
\renewcommand{\cftsecpagefont}{\rmfamily\mdseries\upshape} % No bold!
\captionsetup[figure]{labelfont={bf,it},textfont={it}}
\setlength{\parindent}{0cm}
%%% END Article customizations

%%% The "real" document content comes below...

\title{Probability with martingales - exercises}
\author{}
\date{} % Activate to display a given date or no date (if empty),
         % otherwise the current date is printed 

\begin{document}
{\bf \maketitle \par}


\section*{E10.1}

\textbf{First part}

$\mathcal{F}_{n}:=\sigma\left(B_{1},B_{2},...,B_{n}\right)$

$M_{n}$ is $\mathcal{F}_{n}$-measurable since $M_{n}$ is a one-to-one function of $B_{n}$ and $\sigma\left(B_{n}\right)\subseteq \mathcal{F}_{n}$.

$\mathrm{E}\left(|M_{n}|\right)<\infty$ for all $n$, since  $0 \leq M_{n} \leq 1$ .

\vspace{\baselineskip}
\begin{equation*}
\begin{array}{lll}
\mathrm{E}\left(M_{n}|\mathcal{F}_{n-1}\right)&=&\mathrm{E}\left[M_{n}|\sigma\left(\sigma\left(B_{n-1}\right),\mathcal{F}_{n-2}\right)\right]\\
&=&\mathrm{E}\left[M_{n}|\sigma\left(B_{n-1}\right)\right]\\
&=&\left(\frac{B_{n-1}+1}{n+2}\right)\left(\frac{n-B_{n-1}}{n+1}\right)+\left(\frac{B_{n-1}+2}{n+2}\right)\left(\frac{B_{n-1}+1}{n+1}\right)\\
&=&\left(\frac{B_{n-1}+1}{n+1}\right)\\
&=&M_{n-1} \quad.
\end{array}
\end{equation*}

\textbf{Second part}

Let $\mathbf{P}$ be the one-step transition matrix from time $n$ to time $n+1$ for the martingale $M$. Then,

\begin{equation*}
\renewcommand{\arraystretch}{1.5}
\mathbf{P}=\kbordermatrix{~&\left(\frac{1}{n+3}\right)&\left(\frac{2}{n+3}\right)&\left(\frac{3}{n+3}\right)&\hdots&\left(\frac{n+2}{n+3}\right)\\
				(\frac{1}{n+2})&\frac{n+1}{n+2}&\frac{1}{n+2}&0&\hdots&0\\
				(\frac{2}{n+2})&0&\frac{n}{n+2}&\frac{2}{n+2}&\hdots&0\\
			    \vdots &\vdots&\vdots&\ddots&\ddots&\vdots\\
			    (\frac{n+1}{n+2})&0&0&0&\frac{1}{n+2}&\frac{n+1}{n+2}}.
\end{equation*}

If $\boldsymbol{\mu^{(n)}}=\left[\frac{1}{n+1} \quad
\frac{1}{n+1} \quad \hdots \quad \frac{1}{n+1}\right]$, then
\begin{equation*}
\begin{array}{lll}
 \boldsymbol{\mu^{(n)}}\mathbf{P}&=&\left[\frac{1}{n+2} \quad \frac{1+n}{(n+1)(n+2)}\quad \frac{2+n-1}{(n+1) (n+2)} \quad \hdots\quad\right]\\
 &=& \left[\frac{1}{n+2} \quad \frac{1}{n+2} \quad \hdots \quad \right]\\
&=&\boldsymbol{\mu^{(n+1)}} \quad .
\end{array}
\end{equation*}

In addition, $\boldsymbol{\mu^{(0)}}:=\left[\quad 1 \quad \right]$, so it must be that $\boldsymbol{\mu^{(n)}}=\left[\frac{1}{n+1} \quad
\frac{1}{n+1} \quad \hdots \quad \frac{1}{n+1}\right]$. 

And $\mu^{(n)}_{k}=\mathrm{P}\left(B_{n}=k\right)$.
\vspace{\baselineskip}


$\mathrm{P} \left(M_{n} \leq m\right)= \mathrm{P} \left(\frac{B_{n}+1}{n+2} \leq m \right)=\mathrm{P}\left[ B_{n} \leq m(n+2)-1\right]=\frac{m(n+2)-1}{n+1} \to m$, as $n \to \infty$.

So $M_{n}$ converges in distribution to $\Theta \sim$ U$(0,1)$.  
\vspace{\baselineskip}

\textbf{Third part}
\begin{equation*}
{\setstretch{1.75}
\begin{array}{lll}
\mathrm{E}\left[N^{\theta}_{n}|\sigma \left(N^{\theta}_{n-1},N^{\theta}_{n-2}, ...\right)\right]&=&\mathrm{E}\left[N^{\theta}_{n}|\sigma \left(N^{\theta}_{n-1}\right)\right]\\
&=&\mathrm{E}\left[N^{\theta}_{n}|\sigma \left(B_{n-1}\right)\right]\\
&=&\frac{\left(n+1\right)!}{B_{n-1}!\left(n-B_{n-1}\right)!}\theta^{B_{n-1}}\left(1-\theta\right)^{n-B_{n-1}}\left(\frac{n-B_{n-1}}{n+1}\right)\\
&+&\frac{\left(n+1\right)!}{\left(B_{n-1}+1\right)!\left(n-B_{n-1}-1\right)!}\theta^{B_{n-1}+1}\left(1-\theta\right)^{n-B_{n-1}-1}\left(\frac{B_{n-1}+1}{n+1}\right)\\
&=&...\\
&=&\frac{n!}{B_{n-1}!\left(n-1-B_{n-1}\right)!}\left[\theta^{B_{n-1}}\left(1-\theta\right)^{B_{n-1}}+\theta^{B_{n-1}+1}\left(1-\theta\right)^{n-1-B_{n-1}}\right]\\
&=&\frac{n!}{B_{n-1}!\left(n-1-B_{n-1}\right)!}\theta^{B_{n-1}}\left(1-\theta\right)^{n-1-B_{n-1}}\left[\left(1-\theta\right)+\theta\right]\\
&=&N^{\theta}_{n-1} \quad.
\end{array}
}
\end{equation*}





 



\clearpage
\bibliographystyle{Chicago}
\bibliography{Researchplan}
\end{document}
